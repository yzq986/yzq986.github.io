%-------------------------
% Resume in Latex (Chinese Version)
% Auto-generated from JSON data
% For Overleaf: Set compiler to XeLaTeX in Menu
%------------------------

\documentclass[letterpaper,11pt]{article}

\usepackage{latexsym}
\usepackage[empty]{fullpage}
\usepackage{titlesec}
\usepackage{marvosym}
\usepackage[usenames,dvipsnames]{color}
\usepackage{verbatim}
\usepackage{enumitem}
\usepackage[hidelinks]{hyperref}
\usepackage{tabularx}
\usepackage{fontawesome5}
\usepackage{multicol}
\usepackage{graphicx}
\setlength{\multicolsep}{-3.0pt}
\setlength{\columnsep}{-1pt}
% glyphtounicode is for pdfLaTeX only, not needed for XeLaTeX

\RequirePackage{tikz}
\RequirePackage{xcolor}

% Chinese support - XeLaTeX required
% Using xeCJK package as recommended by Overleaf
\usepackage{xeCJK}
% Default font in Overleaf is Fandol (supports both Simplified and Traditional Chinese)
% Explicitly setting fonts for compatibility
\setCJKmainfont{FandolSong}[
  BoldFont=FandolSong-Bold,
]
\setCJKsansfont{FandolHei}[
  BoldFont=FandolHei-Bold,
]
\setCJKmonofont{FandolFang}

\definecolor{cvblue}{HTML}{0E5484}
\definecolor{black}{HTML}{130810}
\definecolor{darkcolor}{HTML}{0F4539}
\definecolor{cvgreen}{HTML}{3BD80D}
\definecolor{taggreen}{HTML}{00E278}
\definecolor{SlateGrey}{HTML}{2E2E2E}
\definecolor{LightGrey}{HTML}{666666}
\colorlet{name}{black}
\colorlet{tagline}{darkcolor}
\colorlet{heading}{darkcolor}
\colorlet{headingrule}{cvblue}
\colorlet{accent}{darkcolor}
\colorlet{emphasis}{SlateGrey}
\colorlet{body}{LightGrey}

% Adjust margins
\addtolength{\oddsidemargin}{-0.6in}
\addtolength{\evensidemargin}{-0.5in}
\addtolength{\textwidth}{1.19in}
\addtolength{\topmargin}{-.7in}
\addtolength{\textheight}{1.4in}
\urlstyle{same}

\definecolor{airforceblue}{rgb}{0.36, 0.54, 0.66}

\raggedbottom
\raggedright
\setlength{\tabcolsep}{0in}

% Sections formatting
\titleformat{\section}{
  \vspace{-4pt}\scshape\raggedright\large\bfseries
}{}{0em}{}[\color{black}\titlerule \vspace{-5pt}]

% XeLaTeX generates Unicode PDF by default, no need for pdfgentounicode

%-------------------------
% Custom commands
\newcommand{\resumeItem}[1]{
  \item\small{
    {#1 \vspace{-1pt}}
  }
}

\newcommand{\resumeItemNoBullet}[1]{
  \item[]\small{
    {#1 \vspace{-1pt}}
  }
}

\newcommand{\classesList}[4]{
    \item\small{
        {#1 #2 #3 #4 \vspace{-2pt}}
  }
}

\newcommand{\resumeSubheading}[4]{
  \vspace{-2pt}\item
    \begin{tabular*}{1.0\textwidth}[t]{l@{\extracolsep{\fill}}r}
      \textbf{\large#1} & \textbf{\small #2} \\
      \textit{\large#3} & \textit{\small #4} \\

    \end{tabular*}\vspace{-7pt}
}

\newcommand{\resumeSingleSubheading}[4]{
  \vspace{-2pt}\item
    \begin{tabular*}{1.0\textwidth}[t]{l@{\extracolsep{\fill}}r}
      \textbf{\large#1} & \textbf{\small #2} \\

    \end{tabular*}\vspace{-7pt}
}

\newcommand{\resumeSubSubheading}[2]{
    \item
    \begin{tabular*}{0.97\textwidth}{l@{\extracolsep{\fill}}r}
      \textit{\small#1} & \textit{\small #2} \\
    \end{tabular*}\vspace{-7pt}
}

\newcommand{\resumeProjectHeading}[2]{
    \item
    \begin{tabular*}{1.001\textwidth}{l@{\extracolsep{\fill}}r}
      \small#1 & \textbf{\small #2}\\
    \end{tabular*}\vspace{-7pt}
}

\newcommand{\resumeSubItem}[1]{\resumeItem{#1}\vspace{-4pt}}

\renewcommand\labelitemi{$\vcenter{\hbox{\tiny$\bullet$}}$}
\renewcommand\labelitemii{$\circ$}

\newcommand{\resumeSubHeadingListStart}{\begin{itemize}[leftmargin=0.0in, label={}]}
\newcommand{\resumeSubHeadingListEnd}{\end{itemize}}
\newcommand{\resumeItemListStart}{\begin{itemize}[leftmargin=0.1in]}
\newcommand{\resumeItemListEnd}{\end{itemize}\vspace{-5pt}}

\newcommand\sbullet[1][.5]{\mathbin{\vcenter{\hbox{\scalebox{#1}{$\bullet$}}}}}

%-------------------------------------------
%%%%%%  RESUME STARTS HERE  %%%%%%%%%%%%%%%%%%%%%%%%%%%%

\begin{document}
%----------HEADING----------

\begin{center}
    {\huge 叶子卿} \\ \vspace{2pt}
    {(+86) 131-****-6485}
    \small{-}
    \href{mailto:yez****@mail2.sysu.edu.cn}{\color{blue}{yez****@mail2.sysu.edu.cn}}
    \small{-}
    \href{https://yzq986.github.io}{\color{blue}{yzq986.github.io}}
    \vspace{-7pt}
\end{center}

%-----------PROGRAMMING SKILLS-----------
\section{\color{airforceblue}技术能力}
 \begin{itemize}[leftmargin=0in, label={}]
    \small{\item{
     \textbf{\normalsize{算法:}}{ \normalsize{推荐系统架构、模型优化、深度学习}} \\
     \textbf{\normalsize{语言:}}{ \normalsize{C++、Python、Java}} \\
     \textbf{\normalsize{机器学习:}}{ \normalsize{TensorFlow、Pytorch}} \\
     \textbf{\normalsize{数据与工程:}}{ \normalsize{Spark、Hadoop、Docker、Kubernetes、AWS、Git}}
     }}
 \end{itemize}
 \vspace{-16pt}

%-----------AWARDS & HONORS---------------
\section{\color{airforceblue}奖项与荣誉}

  \resumeItemListStart
    \resumeItem{\normalsize{\textbf{金奖} --- 2017年ACM-ICPC国际大学生程序设计竞赛亚洲区域赛总决赛 (2017年12月)}}
    \vspace{-5pt}
    \resumeItem{\normalsize{\textbf{金奖第五名} --- 2017年ACM-ICPC国际大学生程序设计竞赛亚洲区域赛香港站 (2017年11月)}}
    \vspace{-5pt}
    \resumeItem{\normalsize{\textbf{金奖} --- 2017年ACM-ICPC国际大学生程序设计竞赛亚洲区域赛沈阳站 (2017年10月)}}
    \vspace{-5pt}
    \resumeItem{\normalsize{\textbf{金奖} --- 2016年ACM-ICPC国际大学生程序设计竞赛亚洲区域赛青岛站 (2016年11月)}}
    \vspace{-5pt}
    \resumeItem{\normalsize{\textbf{金奖第五名} --- 2016年CCPC中国大学生程序设计竞赛长春赛区 (2016年9月)}}
    \vspace{-5pt}
    \resumeItem{\normalsize{\textbf{银奖第十八名} --- 2016年CCPC中国大学生程序设计竞赛总决赛 (2016年12月)}}
    \vspace{-5pt}
    \resumeItem{\normalsize{\textbf{铜牌} --- 2013年全国青少年信息学奥林匹克竞赛 (2013年8月)}}
    \vspace{-5pt}
    \resumeItem{\normalsize{\textbf{Meritorious Winner} --- 2017年美国大学生数学建模竞赛 (2017年4月)}}
    \vspace{-5pt}
  \resumeItemListEnd

\vspace{-8pt}

%-----------EXPERIENCE-----------
\section{\color{airforceblue}工作经历}
  \resumeSubHeadingListStart

    \resumeSubheading
      {某头部交易所 - 广场Feed流 - 算法负责人}{}
      {广场推荐算法负责人 | Tech Lead}{2025年 -- 至今}
      \resumeItemListStart
            \resumeItem{\normalsize{作为广场Feed流推荐算法负责人,端到端负责推荐系统的算法优化、工程架构和团队协作,推动广场交易转化实现质的飞跃。通过建立完整的交易归因体系、驱动跨团队协作、全栈优化系统性能,年度贡献交易额预估增量41.1亿 USD}}
            \vspace{2pt}
            \resumeItem{\normalsize{\textbf{【Ownership \& Leadership】端到端主导推荐系统架构升级与业务增长}}}
            \begin{itemize}[leftmargin=0.2in]
                \item \normalsize{作为Tech Lead主导推荐系统全链路优化(v23→v24→v25三个版本迭代),协调算法、工程、业务三方团队(10+成员),建立跨部门协作机制}
                \item \normalsize{年度核心成果:组件导流订单量+30.95\%,日均交易额1.26亿 USD(年度增量41.1亿 USD),转化效率提升23.94\%}
                \item \normalsize{建立完整的效果评估体系与AB测试框架,确保所有指标提升均为纯模型贡献,排除产品、运营等其他因素干扰}
            \end{itemize}
            \vspace{1pt}
            \resumeItem{\normalsize{\textbf{【全栈工程能力】系统性能优化与成本控制}}}
            \begin{itemize}[leftmargin=0.2in]
                \item \normalsize{主导离线训练链路优化:发现并解决离线特征构建内存爆炸问题,设计两阶段去重策略,内存峰值-50\%,失败率从30\%降至<1\%,处理时间从50分钟降至30分钟}
                \item \normalsize{主导SageMaker线上部署优化:与工程团队深度协作,将实例配置从300台4x large优化至150台8x large,资源成本降低50\%,保证推理性能和稳定性}
                \item \normalsize{年度成本优化:离线训练成本从日均6000 USD降至3000 USD,线上推理成本-50\%,累计年预估节省200万+ USD}
            \end{itemize}
            \vspace{1pt}
            \resumeItem{\normalsize{\textbf{【跨团队影响力】推动组织协作与技术体系建设}}}
            \begin{itemize}[leftmargin=0.2in]
                \item \normalsize{与交易团队(Tat/Maya/Choco)建立深度合作关系,打通内容推荐与交易转化的数据链路,首次实现内容对交易贡献的精准量化}
                \item \normalsize{推动工程团队(Jeffrey/JerryM)完成特征存储优化(压缩30\%存储成本)、推理性能优化、部署流水线建设,提升团队整体迭代效率}
                \item \normalsize{建立标准化的模型上线SOP、效果评估体系、问题诊断流程,沉淀方法论文档,推动知识传承与团队能力提升}
            \end{itemize}
            \vspace{1pt}
            \resumeItem{\normalsize{\textbf{【持续迭代与深度优化】特征工程与模型架构演进}}}
            \begin{itemize}[leftmargin=0.2in]
                \item \normalsize{v23模型:首次在推荐模型中引入交易目标,从0到1设计完整的交易归因技术方案:梳理20+交易埋点,设计60s/1h窗口期回补机制,引入交易底表覆盖futures/spot/alpha三大场景}
                \item \normalsize{创新性地将trade\_click、place\_order、main\_trade等多个交易目标融入多目标模型,实现点击-时长-交易的端到端优化,解决交易归因准确性问题}
                \item \normalsize{60秒订单成交量+16.07\%,人均订单数+21.78\%;高转化场景人均订单增幅达59\%-76\%}
                \item \normalsize{v24模型:系统性完善特征体系,新增类目特征v3(赛道/产品/内容三维分类),增强用户行为序列特征(click/like/share/follow/comment对应的类目特征及统计特征),60秒人均成单量+32.78\%}
                \item \normalsize{v25模型:完善历史交易组件点击序列特征(新增author/token/zone/keyword等8个side info特征),以序列方式加入模型,提升用户交易意图理解,订单日均+1.08\%}
                \item \normalsize{主动识别并修复多个历史遗留的特征bug,建立完整的数据质量监控体系,保证模型基础质量}
            \end{itemize}
            \vspace{1pt}
      \resumeItemListEnd


    \resumeSubheading
      {Tiktok - 国际化电商 - 算法专家}{}
      {国际化电商推荐组成员}{2024年 -- 2025年}
      \resumeItemListStart
            \resumeItem{\normalsize{排序基础技术优化,持续迭代粗精排的特征信号、模型结构、样本和排序目标}}
            \vspace{2pt}
            \resumeItem{\normalsize{\textbf{粗排 Listwise 数据流蒸馏}}}
            \begin{itemize}[leftmargin=0.2in]
                \item \normalsize{通过补充全精排打分样本至粗排数据流,实现线上线下打分对齐}
                \item \normalsize{采用蒸馏任务作为隐式任务头,优化蒸馏策略,反转掉重型结构}
                \item \normalsize{US地区获得商城人均GMV+1.4957\%,UV\_CTR+0.4300\%显著提升}
            \end{itemize}
            \vspace{1pt}
            \resumeItem{\normalsize{\textbf{粗排融合公式迭代,VT链路重构}}}
            \begin{itemize}[leftmargin=0.2in]
                \item \normalsize{优化粗排融合公式配置化管理,简化模型调整流程}
                \item \normalsize{系统化调参策略优化}
                \item \normalsize{US区域获得显著增长:UV\_CTR +0.3650\%,GMV +0.9536\%}
            \end{itemize}
            \vspace{1pt}
            \resumeItem{\normalsize{\textbf{精排子场景,小模型替代大模型}}}
            \begin{itemize}[leftmargin=0.2in]
                \item \normalsize{通过复用全场景大模型与场景内小模型的结合方式,成功用Lite模型替代原有场景内大模型}
                \item \normalsize{小模型使用场景内样本流训练,减少资源消耗,加速训练过程}
                \item \normalsize{保持GMV和CTCVR基本持平的同时,减少52.73\%训练PS消耗与79.9\%线上PS消耗}
            \end{itemize}
            \vspace{1pt}
      \resumeItemListEnd


    \resumeSubheading
      {Shopee SG - 高级算法工程师}{}
      {Daily Discovery组成员}{2021年 -- 2024年}
      \resumeItemListStart
            \resumeItem{\normalsize{针对Shopee首页推荐业务,设计排序模型以及效果性能优化}}
            \vspace{2pt}
            \resumeItem{\normalsize{\textbf{全区域上线第一版多目标级联粗排模型}}}
            \begin{itemize}[leftmargin=0.2in]
                \item \normalsize{与精排共用训练推理框架,通过缓存item侧embedding加速线上推理过程}
                \item \normalsize{选用三塔+顶部MLP融合架构,在保证推理效率前提下获得交叉信息}
                \item \normalsize{在3000 item输入量下,粗排延迟约46ms,全区域获得显著提升}
                \item \normalsize{ID +2.33\% click +1.59\%,BR +2.3\% order,VN +3.92\% order等多项指标改善}
            \end{itemize}
            \vspace{1pt}
            \resumeItem{\normalsize{\textbf{双粗排联合模型研究实践}}}
            \begin{itemize}[leftmargin=0.2in]
                \item \normalsize{使用in-batch随机负样本并借鉴CLIP的双向计算CE loss与自适应温度系数}
                \item \normalsize{离线实现+5.23\%的order提升,线上ID和SG区域分别取得+2.3\% click和+3.36\% order提升}
            \end{itemize}
            \vspace{1pt}
            \resumeItem{\normalsize{\textbf{粗排多目标升级ESMM结构}}}
            \begin{itemize}[leftmargin=0.2in]
                \item \normalsize{引入多级别目标,灵活调整线上评分序列的权重,优化item排名与隐含目标训练}
                \item \normalsize{ID区域获得order/u 3.35\%的提升}
            \end{itemize}
            \vspace{1pt}
            \resumeItem{\normalsize{\textbf{精排升级JRC结构}}}
            \begin{itemize}[leftmargin=0.2in]
                \item \normalsize{将pointwise训练替换为listwise训练,借鉴RCR loss和JRC论文中的CE+GE loss}
                \item \normalsize{ID区域获得click/u 4\%的提升}
            \end{itemize}
            \vspace{1pt}
            \resumeItem{\normalsize{\textbf{商店展示页精排模型特征筛选}}}
            \begin{itemize}[leftmargin=0.2in]
                \item \normalsize{引入slot乘子优化特征筛选,去掉59个低权重乘子}
                \item \normalsize{减少50\%存储容量,提升50\%训练速度}
                \item \normalsize{全局click持平情况下,order指标提升+1.27\%}
            \end{itemize}
            \vspace{1pt}
      \resumeItemListEnd


    \resumeSubheading
      {微信视频号 - 算法工程师}{}
      {算法推荐一组成员}{2020年 -- 2021年}
      \resumeItemListStart
            \resumeItem{\normalsize{完成微信视频号视频推荐算法相关研究及后台数据支持}}
            \vspace{2pt}
            \resumeItem{\normalsize{\textbf{lookalike模型架构搭建}}}
            \begin{itemize}[leftmargin=0.2in]
                \item \normalsize{在热门feed推荐中搭建lookalike链路,完成u2u人群扩展}
                \item \normalsize{通过mq缓冲处理,使同一item的曝光聚合后生成hist数据,日曝光量约3.3亿}
                \item \normalsize{使用cache做轮次内缓存,bdemem做轮次间缓存,优化线上推理性能}
            \end{itemize}
            \vspace{1pt}
            \resumeItem{\normalsize{\textbf{框架预测流程迁移接入metis 1.0架构}}}
            \begin{itemize}[leftmargin=0.2in]
                \item \normalsize{完成模型架构转移,使模型infer embedding部分逻辑与原流程解耦}
                \item \normalsize{新建rpc接口,负责对所有请求计算infer的embedding值}
            \end{itemize}
            \vspace{1pt}
            \resumeItem{\normalsize{\textbf{冷启动推荐提权融合公式修改}}}
            \begin{itemize}[leftmargin=0.2in]
                \item \normalsize{替换原有冷启链路,启用基于pid控制的融合提权公式}
                \item \normalsize{精细化考虑观看时长、完播、点赞等特征,调整冷启出量控制}
            \end{itemize}
            \vspace{1pt}
            \resumeItem{\normalsize{\textbf{推荐架构数组类型特征支持}}}
            \begin{itemize}[leftmargin=0.2in]
                \item \normalsize{在视频feed推荐流程中新增对数组类型feature的支持}
                \item \normalsize{完成在数据流模型中各部分对列表特征的支持}
                \item \normalsize{完成在线数据流与模型训练的对接工作}
            \end{itemize}
            \vspace{1pt}
      \resumeItemListEnd


    \resumeSubheading
      {Google - 软件开发工程师}{}
      {软件开发工程师}{2019年 -- 2020年}
      \resumeItemListStart
            \resumeItem{\normalsize{完成广告收入预测项目的开发、实现、测试、上线}}
            \vspace{2pt}
            \resumeItem{\normalsize{\textbf{广告收入预测项目}}}
            \begin{itemize}[leftmargin=0.2in]
                \item \normalsize{基于Google广告业务的收入数据,设计算法框架,实现对广告收入的预测}
                \item \normalsize{分别对一天、一个月、一个季度、一年的收入进行预测}
                \item \normalsize{识别每年的高峰收入时间区域,并完成对比测试和上线}
            \end{itemize}
            \vspace{1pt}
      \resumeItemListEnd


    \resumeSubheading
      {BIGO - 算法工程师 - 实习}{}
      {算法组成员}{2019年}
      \resumeItemListStart
            \resumeItem{\normalsize{完成音频快速聚类算法专利,实现所有代码细节}}
            \vspace{2pt}
            \resumeItem{\normalsize{\textbf{音频快速聚类算法专利}}}
            \begin{itemize}[leftmargin=0.2in]
                \item \normalsize{针对BIGO短视频app Like的音频特点,设计适合在给定时间限制内完成大量短视频音频比对的hash算法}
                \item \normalsize{设计对应的特征聚类算法,针对业务特点进行算法时间复杂度优化}
                \item \normalsize{使算法适用于业务场景大规模的数据比对需求}
            \end{itemize}
            \vspace{1pt}
      \resumeItemListEnd


    \resumeSubheading
      {香港理工大学 - 研究助理}{}
      {研究助理(交换生)}{2018年}
      \resumeItemListStart
            \resumeItem{\normalsize{作为交换生期间担任研究助理,主要研究区块链技术在供应链管理中的应用}}
            \vspace{2pt}
            \resumeItem{\normalsize{\textbf{区块链驱动的可信供应链系统研究}}}
            \begin{itemize}[leftmargin=0.2in]
                \item \normalsize{研究区块链技术在供应链可信度和透明度方面的应用潜力}
                \item \normalsize{设计并实现基于区块链的供应链追踪原型系统}
            \end{itemize}
            \vspace{1pt}
      \resumeItemListEnd

  \resumeSubHeadingListEnd
\vspace{-12pt}

%-----------EDUCATION-----------
\section{\color{airforceblue}教育背景}
  \resumeSubHeadingListStart
    \resumeSubheading
      {中山大学}{2015 年 -- 2019 年}
      {计算机科学与技术专业}{GPA: 3.8/4.0}
  \resumeSubHeadingListEnd
  \vspace{-10pt}

%-----------EXTRACURRICULAR---------------
\section{\color{airforceblue}课外活动}

      \resumeItemListStart
        \resumeItem{\normalsize{\textbf{技术部部长} --- 中山大学ACMM协会 --- {2016年 -- 2017年} }}
        \resumeItem{\normalsize{组织举办算法讲座、程序设计新手赛、校赛}}
        \vspace{-5pt}
      \resumeItemListEnd

\vspace{-12pt}

\end{document}